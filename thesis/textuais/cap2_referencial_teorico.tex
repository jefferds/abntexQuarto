% Options for packages loaded elsewhere
% Options for packages loaded elsewhere
\PassOptionsToPackage{unicode}{hyperref}
\PassOptionsToPackage{hyphens}{url}
\PassOptionsToPackage{dvipsnames,svgnames,x11names}{xcolor}
%
\documentclass[
  12pt,
  oneside,
  a4paper,
  english,
  brazil]{abntex2}
\usepackage{xcolor}
\usepackage[top=30mm,left=30mm,right=20mm,bottom=20mm]{geometry}
\usepackage{amsmath,amssymb}
\setcounter{secnumdepth}{5}
\usepackage{iftex}
\ifPDFTeX
  \usepackage[T1]{fontenc}
  \usepackage[utf8]{inputenc}
  \usepackage{textcomp} % provide euro and other symbols
\else % if luatex or xetex
  \usepackage{unicode-math} % this also loads fontspec
  \defaultfontfeatures{Scale=MatchLowercase}
  \defaultfontfeatures[\rmfamily]{Ligatures=TeX,Scale=1}
\fi
\usepackage{lmodern}
\ifPDFTeX\else
  % xetex/luatex font selection
\fi
% Use upquote if available, for straight quotes in verbatim environments
\IfFileExists{upquote.sty}{\usepackage{upquote}}{}
\IfFileExists{microtype.sty}{% use microtype if available
  \usepackage[]{microtype}
  \UseMicrotypeSet[protrusion]{basicmath} % disable protrusion for tt fonts
}{}
\makeatletter
\@ifundefined{KOMAClassName}{% if non-KOMA class
  \IfFileExists{parskip.sty}{%
    \usepackage{parskip}
  }{% else
    \setlength{\parindent}{0pt}
    \setlength{\parskip}{6pt plus 2pt minus 1pt}}
}{% if KOMA class
  \KOMAoptions{parskip=half}}
\makeatother
% Make \paragraph and \subparagraph free-standing
\makeatletter
\ifx\paragraph\undefined\else
  \let\oldparagraph\paragraph
  \renewcommand{\paragraph}{
    \@ifstar
      \xxxParagraphStar
      \xxxParagraphNoStar
  }
  \newcommand{\xxxParagraphStar}[1]{\oldparagraph*{#1}\mbox{}}
  \newcommand{\xxxParagraphNoStar}[1]{\oldparagraph{#1}\mbox{}}
\fi
\ifx\subparagraph\undefined\else
  \let\oldsubparagraph\subparagraph
  \renewcommand{\subparagraph}{
    \@ifstar
      \xxxSubParagraphStar
      \xxxSubParagraphNoStar
  }
  \newcommand{\xxxSubParagraphStar}[1]{\oldsubparagraph*{#1}\mbox{}}
  \newcommand{\xxxSubParagraphNoStar}[1]{\oldsubparagraph{#1}\mbox{}}
\fi
\makeatother


\usepackage{longtable,booktabs,array}
\usepackage{calc} % for calculating minipage widths
% Correct order of tables after \paragraph or \subparagraph
\usepackage{etoolbox}
\makeatletter
\patchcmd\longtable{\par}{\if@noskipsec\mbox{}\fi\par}{}{}
\makeatother
% Allow footnotes in longtable head/foot
\IfFileExists{footnotehyper.sty}{\usepackage{footnotehyper}}{\usepackage{footnote}}
\makesavenoteenv{longtable}
\usepackage{graphicx}
\makeatletter
\newsavebox\pandoc@box
\newcommand*\pandocbounded[1]{% scales image to fit in text height/width
  \sbox\pandoc@box{#1}%
  \Gscale@div\@tempa{\textheight}{\dimexpr\ht\pandoc@box+\dp\pandoc@box\relax}%
  \Gscale@div\@tempb{\linewidth}{\wd\pandoc@box}%
  \ifdim\@tempb\p@<\@tempa\p@\let\@tempa\@tempb\fi% select the smaller of both
  \ifdim\@tempa\p@<\p@\scalebox{\@tempa}{\usebox\pandoc@box}%
  \else\usebox{\pandoc@box}%
  \fi%
}
% Set default figure placement to htbp
\def\fps@figure{htbp}
\makeatother





\setlength{\emergencystretch}{3em} % prevent overfull lines

\providecommand{\tightlist}{%
  \setlength{\itemsep}{0pt}\setlength{\parskip}{0pt}}



 
\usepackage[]{natbib}
\bibliographystyle{plainnat}


% ---
% Pacotes básicos 
% ---
\usepackage{lmodern}			% Usa a fonte Latin Modern			
\usepackage[T1]{fontenc}		% Selecao de codigos de fonte.
\usepackage[utf8]{inputenc}		% Codificacao do documento (conversão automática dos acentos)
\usepackage{lastpage}			% Usado pela Ficha catalográfica
\usepackage{indentfirst}		% Indenta o primeiro parágrafo de cada seção.
\usepackage{color}				% Controle das cores
\usepackage{graphicx}			% Inclusão de gráficos
\usepackage{microtype} 			% para melhorias de justificação
\usepackage{amsmath}
\usepackage{amssymb}
\usepackage{upgreek}
\usepackage{listings}			% Inclusão de código fonte 			


% Pacotes de citações
% ---
\usepackage[brazilian,hyperpageref]{backref}	 % Paginas com as citações na bibl
\usepackage[alf]{abntex2cite}	% Citações padrão ABNT

% --- 
% CONFIGURAÇÕES DE PACOTES
% --- 

% ---
% Configurações do pacote backref
% Usado sem a opção hyperpageref de backref
\renewcommand{\backrefpagesname}{Citado na(s) página(s):~}
% Texto padrão antes do número das páginas
\renewcommand{\backref}{}
% Define os textos da citação
\renewcommand*{\backrefalt}[4]{
	\ifcase #1 %
		Nenhuma citação no texto.%
	\or
		Citado na página #2.%
	\else
		Citado #1 vezes nas páginas #2.%
	\fi}%
% ---

% ---
% Configurações de aparência do PDF final

% alterando o aspecto da cor azul
\definecolor{blue}{RGB}{41,5,195}

% informações do PDF
\makeatletter
\hypersetup{
     	%pagebackref=true,
		% pdftitle={\@title}, 
		% pdfauthor={\@author},
    	% pdfsubject={\imprimirpreambulo},
	    % pdfcreator={LaTeX with abnTeX2},
		% pdfkeywords={abnt}{latex}{abntex}{abntex2}{trabalho acadêmico}, 
		colorlinks=true,       		% false: boxed links; true: colored links
    	linkcolor=blue,          	% color of internal links
    	citecolor=blue,        		% color of links to bibliography
    	filecolor=magenta,      		% color of file links
		urlcolor=blue,
		bookmarksdepth=4
}

% Comandos de uso do orientador para sugestão de alterações:
\usepackage{xcolor}
\usepackage{cancel}

\colorlet{oliveGreen}{blue!20!black!50!green}

\usepackage{soul}
\usepackage{soulutf8}
\setstcolor{red}

\newcommand{\cito}[1]{\textsuperscript{\cite{#1}}} % numeração da citação sobreescrita
\newcommand{\PROFESSOR}[1]{\textcolor{red}{\MakeUppercase{#1}}} % Acrescenta um nota pelo professor
\newcommand{\PROFESSORDEL}[1]{\textcolor{red}{\st{#1}}} % removido pelo professor
\newcommand{\PROFESSORADD}[1]{\textcolor{oliveGreen}{#1}} % adicionado pelo professor
\newcommand{\PROFESSORREP}[2]{\PROFESSORDEL{#1}\PROFESSORADD{#2}} % substituir


\renewcommand{\printtoctitle}[1]{\normalsize\bfseries\centering TABLE OF CONTENTS}


% Adiciona indentação para seções e subseções no sumário
\cftsetindents{chapter}{0em}{0em}
\cftsetindents{section}{1.5em}{2.5em}
\cftsetindents{subsection}{3em}{3.5em}
\cftsetindents{subsubsection}{4.5em}{5em}

\AtBeginDocument{\captionsetup{labelfont=bf}}


\makeatother
% --- 

% --- 
% Espaçamentos entre linhas e parágrafos 
% --- 

% O tamanho do parágrafo é dado por:
\setlength{\parindent}{1.3cm}

% Controle do espaçamento entre um parágrafo e outro:
\setlength{\parskip}{0.2cm}  % tente também \onelineskip

% ---
% compila o indice
% ---
\makeindex
% ---
\makeatletter
\@ifpackageloaded{caption}{}{\usepackage{caption}}
\AtBeginDocument{%
\ifdefined\contentsname
  \renewcommand*\contentsname{Table of contents}
\else
  \newcommand\contentsname{Table of contents}
\fi
\ifdefined\listfigurename
  \renewcommand*\listfigurename{List of Figures}
\else
  \newcommand\listfigurename{List of Figures}
\fi
\ifdefined\listtablename
  \renewcommand*\listtablename{List of Tables}
\else
  \newcommand\listtablename{List of Tables}
\fi
\ifdefined\figurename
  \renewcommand*\figurename{Figure}
\else
  \newcommand\figurename{Figure}
\fi
\ifdefined\tablename
  \renewcommand*\tablename{Table}
\else
  \newcommand\tablename{Table}
\fi
}
\@ifpackageloaded{float}{}{\usepackage{float}}
\floatstyle{ruled}
\@ifundefined{c@chapter}{\newfloat{codelisting}{h}{lop}}{\newfloat{codelisting}{h}{lop}[chapter]}
\floatname{codelisting}{Code}
\newcommand*\listoflistings{\listof{codelisting}{List of Code}}
\makeatother
\makeatletter
\makeatother
\makeatletter
\@ifpackageloaded{caption}{}{\usepackage{caption}}
\@ifpackageloaded{subcaption}{}{\usepackage{subcaption}}
\makeatother
\makeatletter
\@ifpackageloaded{tcolorbox}{}{\usepackage[skins,breakable]{tcolorbox}}
\makeatother
\makeatletter
\@ifundefined{shadecolor}{\definecolor{shadecolor}{rgb}{.97, .97, .97}}{}
\makeatother
\makeatletter
\makeatother
\makeatletter
\ifdefined\Shaded\renewenvironment{Shaded}{\begin{tcolorbox}[borderline west={3pt}{0pt}{shadecolor}, boxrule=0pt, sharp corners, interior hidden, enhanced, frame hidden, breakable]}{\end{tcolorbox}}\fi
\makeatother
\usepackage{bookmark}
\IfFileExists{xurl.sty}{\usepackage{xurl}}{} % add URL line breaks if available
\urlstyle{same}
\hypersetup{
  colorlinks=true,
  linkcolor={blue},
  filecolor={Maroon},
  citecolor={Blue},
  urlcolor={Blue},
  pdfcreator={LaTeX via pandoc}}


\author{}
\date{}
\begin{document}


\chapter{\textbf{Literature Review}}

\section{\texorpdfstring{\textbf{Gas Turbines}}{}}\label{section}

\subsection{Gas Turbine: Overview}\label{gas-turbine-overview}

Gas turbines stand as fundamental components within a broad spectrum of
critical applications, ranging from large-scale power generation to the
propulsion of aircraft. Their widespread use underscores the imperative
for precise modeling of their operational behavior
\citet{boyce2012gasturbine}. Such accurate modeling is not merely a
theoretical exercise but is essential for achieving optimal performance,
ensuring operational safety protocols, and successfully implementing
proactive predictive maintenance strategies \citet{boyce2012gasturbine}.
These complex machines operate on fundamental thermodynamic principles,
involving several main components that work in tandem to convert fuel
energy into useful work \citet{cengel2019thermodynamics}.

These core components typically include a compressor, which draws in and
pressurizes air; a combustion chamber, where fuel is mixed with the
compressed air and ignited; a turbine, which extracts energy from the
hot, high-pressure combustion gases; and a nozzle, which accelerates the
exhaust gases to produce thrust or direct them for other purposes
\citet{saravanamuttoo2017gasturbine}. Understanding the intricate
interplay between these components and their adherence to fundamental
physical laws is crucial for their effective design, analysis, and
operation \citet{saravanamuttoo2017gasturbine}. The fundamental
principles governing the design and operation of gas turbine components,
including detailed thermodynamic cycles and performance characteristics,
are extensively documented in specialized literature
\citet{saravanamuttoo2017gasturbine}.

\subsection{Traditional Modeling Approaches and Their
Limitations}\label{traditional-modeling-approaches-and-their-limitations}

Traditional modeling approaches for gas turbines present a significant
trade-off in terms of computational resources and physical consistency.
High-fidelity simulations, often based on first principles such as
computational fluid dynamics (CFD), are known to be computationally
intensive and time-consuming \citet{verstraete2010cfd}. This
characteristic makes them impractical for real-time analysis and control
applications \citet{verstraete2010cfd}. Conversely, purely data-driven
models offer computational efficiency but frequently lack physical
consistency \citet{gurney2010gasturbine}. Such models may also yield
unreliable predictions when extrapolated beyond the dataset they were
trained on \citet{gurney2010gasturbine}. This inherent limitation in
traditional methods highlights a crucial gap, emphasizing the need for a
new generation of models capable of bridging the divide between physical
fidelity and computational efficiency \citet{kurz2009gasturbine}.

\section{\texorpdfstring{\textbf{Foundations for Advanced System Representation}}{}}\label{section-1}

\subsection{Digital Twins Technology}\label{digital-twins-technology}

Digital Twin Technology represents a paradigm shift in system modeling
and management, moving beyond traditional simulation to create a
dynamic, virtual replica of a physical system or process
\citet{grieves2011digital}. This virtual counterpart is continuously
updated with real-time data from its physical twin, allowing for
high-fidelity mirroring of the physical entity's state, behavior, and
performance throughout its lifecycle \citet{tao2019digital}. The core
concept involves the seamless integration of physical and virtual
worlds, enabling predictive analytics, proactive maintenance, and
optimization strategies that were previously unattainable. For complex
systems like gas turbines, a digital twin provides an invaluable tool
for monitoring operational parameters, diagnosing anomalies, and even
predicting future performance degradation. This capability allows
operators to make informed decisions, optimize efficiency, and extend
the lifespan of costly assets, by running simulations and analyses on
the virtual model that directly reflect the real-world conditions
\citet{schluse2018digital}. The utility of digital twins extends across
various stages, from design and manufacturing to operation and
decommissioning, offering a comprehensive and integrated approach to
system representation and control.

\subsection{Artificial Intelligence}\label{artificial-intelligence}

Artificial Intelligence (AI) represents a broad field of computer
science dedicated to creating intelligent agents capable of performing
tasks that typically require human intelligence. These tasks include
learning, problem-solving, perception, and decision-making
\citet{russell2010artificial}. In the context of advanced system
representation, AI plays an important role, particularly through its
subfields such as machine learning and neural networks. Machine learning
algorithms enable systems to learn patterns and make predictions from
data without being explicitly programmed, which is crucial for handling
complex, non-linear relationships often found in engineering systems.
The integration of AI capabilities allows for enhanced predictive power,
adaptive behavior, and the ability to extract insights from vast amounts
of data, significantly contributing to the development of sophisticated
models like digital twins \citet{kreuzer2024artificial}.

Among the various machine learning techniques, Neural Networks (NNs) are
particularly prominent due to their ability to model complex, non-linear
relationships and learn from large datasets. Inspired by the structure
and function of the human brain, NNs consist of interconnected layers of
nodes (neurons) that process information through weighted connections
\citet{goodfellow2016deep}. Their capacity for pattern recognition and
function approximation makes them highly effective for tasks such as
prediction, classification, and control in engineering applications. The
adaptability of NNs allows them to capture intricate dynamics within the
system, making them a powerful tool for building data-driven components
of hybrid models and digital twins.

\section{\texorpdfstring{\textbf{Hybrid Modeling with Physics-Informed Machine Learning}}{}}\label{section-2}

\subsection{The Role of First-Principle
Models}\label{the-role-of-first-principle-models}

First-principle models, grounded in fundamental physical laws such as
thermodynamics, fluid dynamics, and mechanics, serve as the bedrock for
understanding and predicting the behavior of complex engineering systems
like gas turbines \citet{incropera2007fundamentals}. These models
provide inherent physical consistency and interpretability, as their
predictions are directly derived from established scientific principles
rather than solely from observed data. They offer a strong foundation
for analysis, enabling accurate predictions even outside the range of
typical operating conditions, which is a significant advantage over
purely empirical approaches. Furthermore, first-principle models can
capture the underlying mechanisms driving system behavior, offering deep
insights into component interactions and overall performance
\citet{serway2018physics}. However, their development often involves
intricate mathematical formulations and can be computationally
expensive, particularly for high-fidelity simulations of complex
geometries or transient phenomena. This computational burden can limit
their utility for real-time applications or scenarios requiring rapid
iteration.

\subsection{Integrating Physics and Data with
PINNs}\label{integrating-physics-and-data-with-pinns}

Physics-Informed Neural Networks (PINNs) emerge as a powerful solution
to overcome the limitations of both purely data-driven and purely
physics-based models by integrating physical laws directly into the
machine learning framework \citet{raissi2019physics}. This hybrid
approach leverages the universal approximation capabilities of neural
networks to learn from experimental data, while simultaneously enforcing
adherence to the governing physical equations of the system. In PINNs,
the neural network is trained not only to minimize the error between its
predictions and observed data points but also to satisfy the underlying
partial differential equations (PDEs), ordinary differential equations
(ODEs), or algebraic equations that describe the system's physics. This
is achieved by incorporating the residuals of these physics equations
into the network's loss function. The result is a model that is both
data-driven and physically consistent, leading to enhanced predictive
accuracy, improved generalization to unseen conditions, and the ability
to handle sparse or noisy data more effectively
\citet{karniadakis2021physics}.

For gas turbines, PINNs offer a promising avenue for creating robust
digital twins that can accurately predict performance, diagnose faults,
and optimize operations while respecting fundamental thermodynamic and
fluid dynamic principles. A study by \citet{wang2023physics}
demonstrates how PINNs can be effectively employed to model steam
turbine performance, enabling more accurate predictions and condition
monitoring essential for digital twin functionalities. This approach
addresses the computational intensity of traditional physics-based
models while ensuring the physical consistency often lacking in purely
data-driven methods, leading to reliable insights for predictive
maintenance and operational optimization.


\bibliography{../references.bib}



\end{document}
