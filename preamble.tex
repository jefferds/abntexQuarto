% ---
% Pacotes básicos 
% ---
\usepackage{lmodern}			% Usa a fonte Latin Modern			
\usepackage[T1]{fontenc}		% Selecao de codigos de fonte.
\usepackage[utf8]{inputenc}		% Codificacao do documento (conversão automática dos acentos)
\usepackage{lastpage}			% Usado pela Ficha catalográfica
\usepackage{indentfirst}		% Indenta o primeiro parágrafo de cada seção.
\usepackage{color}				% Controle das cores
\usepackage{graphicx}			% Inclusão de gráficos
\usepackage{microtype} 			% para melhorias de justificação
\usepackage{listings}			% Inclusão de código fonte 			


% Pacotes de citações
% ---
\usepackage[brazilian,hyperpageref]{backref}	 % Paginas com as citações na bibl
\usepackage[alf]{abntex2cite}	% Citações padrão ABNT

% --- 
% CONFIGURAÇÕES DE PACOTES
% --- 

% ---
% Configurações do pacote backref
% Usado sem a opção hyperpageref de backref
\renewcommand{\backrefpagesname}{Citado na(s) página(s):~}
% Texto padrão antes do número das páginas
\renewcommand{\backref}{}
% Define os textos da citação
\renewcommand*{\backrefalt}[4]{
	\ifcase #1 %
		Nenhuma citação no texto.%
	\or
		Citado na página #2.%
	\else
		Citado #1 vezes nas páginas #2.%
	\fi}%
% ---

% ---
% Configurações de aparência do PDF final

% alterando o aspecto da cor azul
\definecolor{blue}{RGB}{41,5,195}

% informações do PDF
\makeatletter
\hypersetup{
     	%pagebackref=true,
		% pdftitle={\@title}, 
		% pdfauthor={\@author},
    	% pdfsubject={\imprimirpreambulo},
	    % pdfcreator={LaTeX with abnTeX2},
		% pdfkeywords={abnt}{latex}{abntex}{abntex2}{trabalho acadêmico}, 
		colorlinks=true,       		% false: boxed links; true: colored links
    	linkcolor=blue,          	% color of internal links
    	citecolor=blue,        		% color of links to bibliography
    	filecolor=magenta,      		% color of file links
		urlcolor=blue,
		bookmarksdepth=4
}

% Comandos de uso do orientador para sugestão de alterações:
\usepackage{xcolor}
\usepackage{cancel}

\colorlet{oliveGreen}{blue!20!black!50!green}

\usepackage{soul}
\usepackage{soulutf8}
\setstcolor{red}

\newcommand{\cito}[1]{\textsuperscript{\cite{#1}}} % numeração da citação sobreescrita
\newcommand{\PROFESSOR}[1]{\textcolor{red}{\MakeUppercase{#1}}} % Acrescenta um nota pelo professor
\newcommand{\PROFESSORDEL}[1]{\textcolor{red}{\st{#1}}} % removido pelo professor
\newcommand{\PROFESSORADD}[1]{\textcolor{oliveGreen}{#1}} % adicionado pelo professor
\newcommand{\PROFESSORREP}[2]{\PROFESSORDEL{#1}\PROFESSORADD{#2}} % substituir


\renewcommand{\printtoctitle}[1]{\ABNTEXchapterfontsize\centering SUMÁRIO}

\AtBeginDocument{\captionsetup{labelfont=bf}}


\makeatother
% --- 

% --- 
% Espaçamentos entre linhas e parágrafos 
% --- 

% O tamanho do parágrafo é dado por:
\setlength{\parindent}{1.3cm}

% Controle do espaçamento entre um parágrafo e outro:
\setlength{\parskip}{0.2cm}  % tente também \onelineskip

% ---
% compila o indice
% ---
\makeindex
% ---